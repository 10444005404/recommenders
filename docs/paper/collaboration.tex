\subsection{Collaborative development paradigm} %NOT SURE ABOUT THE TITLE
Majority of the code assets in Microsoft Recommenders are hosted on Github (codes used for the testing pipeline are put on the internal cloud service for maintenance with convenience). The whole life cycle of its development is transparent to the open source community. Codes and notebooks examples were contributed by developers, researchers, and data scientists from industrial organizations and academic research institutes. Such collaborative paradigm is supported mainly by 1) the unified CI/CD build pipeline that runs unit/smoke/integration tests and nightly builds to guarantee the adoption of high-quality codes, 2) the coding principles that standardize the conventions applied in the notebook examples as well as the utility functions. The ease of collaboration in Microsoft Recommenders is applicable for the enteprise-grade development as well. With the help of the best practice examples that cover the entire spectrum in the end-to-end pipeline of a recommender architecture, data scientists, developers, and researchers can harmoniously work out a minimum viable product, which can be easily transformed into productionized pipeline by scaling up or out the underlying computing/data storing infrastruture and build pipeline. 