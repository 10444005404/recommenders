\subsection{Collaborative Development Paradigm} %NOT SURE ABOUT THE TITLE

The majority of the code assets in Microsoft Recommenders are hosted on Github, which offers the advantage that the whole life cycle of its development is transparent to the open source community. This collaborative paradigm is supported mainly by 1) \textit{the unified CI/CD build pipeline that runs unit/smoke/integration tests and nightly builds to guarantee quality of code}, 2) \textit{coding principles that standardize the conventions applied in the notebook examples as well as the utility functions}. For instance, when contributing a new asset (either an example notebook or a utility function) to the repository, one takes the following steps:
\begin{enumerate}
    \item Clone the repository to a local system and set up the development environment.
    \item Develop notebooks and/or utility functions in a new branch that is based on the ``staging'' branch, push the new code to the remote repository, and create a pull request for merging the code to the "staging" branch.
    \item Contributors are requested to follow the test-driven design principle so that unit tests for the notebooks/utilities should be developed together with the submitted code. The newly committed code in the pull request is executed through a set of tests in the CI/CD pipeline hosted on the cloud service, to guarantee that they run without failures.
    \item All contributors are welcome to participate in code reviews, which may address any issues not captured by testing 
    (for example, coding style as described in the coding guidelines\footref{foot_code_guidelines}).  
    \item The code is then merged if all the tests pass and review comments are resolved.
\end{enumerate}

The ease of collaboration in Microsoft Recommenders is extensible to enteprise-grade development. With examples of best practices covering the entire end-to-end pipeline of a recommender architecture, data scientists, developers, and researchers can seamlessly work out a minimum viable product, and transform it into a production-ready system by scaling up the underlying computing/data storing infrastruture and build pipeline. For instance, the combination of versioning and CI/CD tools on a single node machine can be extended to industry-level production systems like Databricks \footnote{\url{https://databricks.com/}} and Azure DevOps \footnote{\url{https://azure.microsoft.com/en-in/services/devops/}} for production use.