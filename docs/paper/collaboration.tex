\subsection{Collaborative Development Paradigm} %NOT SURE ABOUT THE TITLE
Majority of the code assets in Microsoft Recommenders are hosted on Github. The whole life cycle of its development is transparent to the open source community. Codes and notebooks examples were contributed by developers, researchers, and data scientists from industrial organizations and academic research institutes. Such collaborative paradigm is supported mainly by 1) \textit{the unified CI/CD build pipeline that runs unit/smoke/integration tests and nightly builds to guarantee the adoption of high-quality codes}, 2) \textit{the coding principles that standardize the conventions applied in the notebook examples as well as the utility functions}. For instance, in contributing a new asset (either an example notebook or a utility function) to the repository, one takes the following steps
\begin{enumerate}
    \item Clone the repository to a local system and set up the development environment
    \item Develop notebooks and/or utility functions in a new branch that is based on the ``staging'' branch, push the new codes to the remote repository, and create a pull request for merging the codes to the "staging" branch
    \item The code committers are requested to follow the test-driven design principle so that unit tests for the notebooks/utilities should be developed together with the submitted codes. The freshly committed codes in the pull request are executed through a set of testings in the CI/CD pipeline hosted by using the cloud service, to guarantee that they run without failures.
    \item All participants of the repository are allowed to take part in the code review that may not be effectively captured by testing, (e.g., coding style checking as aforementioned in the section of coding guidelines\footref{foot_code_guidelines}.  
    \item The codes are merged if all the testings pass and review comments are resolved
\end{enumerate}

The ease of collaboration in Microsoft Recommenders is extendable to the enteprise-grade development. With the help of the best practice examples that cover the entire spectrum in the end-to-end pipeline of a recommender architecture, data scientists, developers, and researchers can harmoniously work out a minimum viable product, which can be easily transformed into productionized pipeline by scaling up or out the underlying computing/data storing infrastruture and build pipeline. For instance, the combination of versioning tool like Github and CI/CD pipeline on a single node machine can be extended to industry-level production system like Databricks \footnote{\url{https://databricks.com/}} and Azure DevOps \footnote{\url{https://azure.microsoft.com/en-in/services/devops/}} for production use.