\section{Introduction}

Recommendation systems have become ubiquitous in modern business, 
%and have attracted a great deal of interest in academic research.
%According to McKinsey \& Co,
%“35\% of what consumers purchase on Amazon and 75\% of what they watch on Netflix come from recommendations algorithms" 
%\cite{mckinsey}.
%Recommendations appear almost everywhere in today's e-commerce, in which consumers typically have a large variety of products or other
%options to choose from. 
examples of which include recommendation of brands, news, consumer products, media content, 
travel packages and many others. Goals of such scenarios are usually to increase {\em revenue},
{\em customer engagement and satisfaction}, make better {\em predictions}, 
%so that better business decisions about advertising spend and internet traffic can be made. In addition, vendors want to 
gain {\em understanding about customers} etc.
%or segments of customers, in order to address them with more personalized and suitable marketing campaigns.
In machine learning research, the literature on recommendations goes back to the 1990s \cite{Tapestry, grouplens}
and includes a wide variety of approaches, such as {\em collaborative filtering} \cite{bell_lessons,koren2009matrix,SVD++,PMF}, 
{\em content-based} and {\em hybrid} methods \cite{rendle,ffm,bpr,pairwise,multiverse}, 
{\em deep learning} methods \cite{karatzoglou,PNN,cheng2016wide,lian2018xdeepfm,he2017neural,youtube,nvidia,survival},
 {\em explainable recommendation} \cite{explainable,rl_explainable}, {\em reinforcement learning} \cite{rl_explainable,rl,rl_negative} etc.

%The history of recommendation algorithms goes back to the 1990s \cite{Tapestry, grouplens}
%when the first {\em content-based and collaborative filtering} methods appeared.
%Around 2006, the Netflix prize \cite{netflix} provided an additional boost to research in the area, in particular leading to significant advances in factorization-based models \cite{bell_lessons,koren2009matrix,SVD++,PMF}.
%Later on, more {\em hybrid methods} incorporating standard supervised learning into recommendations as well as more hybrid factorization models have appeared
%\cite{rendle,ffm,bpr,pairwise,multiverse}.
%In more recent years, {\em deep learning} methods, after their success in other areas of machine learning, have been applied increasingly in the area of recommendations
%\cite{PNN,cheng2016wide,lian2018xdeepfm,he2017neural,youtube,nvidia,survival}.
%Moreover, the objectives in recommendation problems are broader than, say, prediction accuracy or precision of recommendations, and include a diverse set of tasks and goals such as 
%{\em explainable recommendation} \cite{explainable,rl_explainable}, corrections for {\em biases} and offline {\em evaluation} \cite{mnar,offline,debiasing,counterfactual},
%{\em reinforcement learning} \cite{rl_explainable,rl,rl_negative} etc.

In practice, however, application of recommendation algorithms has encountered significant challenges. 
First, there are {\em limited references and guidance} for building recommendation systems at scale to support
enterprise grade scenarios. In addition, even though there exist several off-the-shelf packages, tools and 
modules \cite{mymedia,Surprise,spotlight,caserec,lenskit}, 
they tend to focus on specific aspects of recommendations, 
cover mostly factorization and neighbor-based methods 
and may {\em not be compatible} with each other. Whereas new algorithms are being published continuously in research, in the field many practitioners may lack the 
expertise or time to implement and deploy these algorithms in production. 
On the other side, researchers, when applying their methods to real world scenarios, may lack experience of the domain of interest
or awareness of the best practices with respect to data science and software engineering. Thus, it can be {\em time-consuming} to build a new recommendation system from an
algorithm even when it is available as software, or to incorporate the algorithm into an end-to-end pipeline. In addition, it is often laborious to prototype new or existing algorithm and fairly compare it against others under the same evaluation protocol. 

In response to these challenges, we developed the recommendation best-practice hub, \verb|Microsoft Recommenders|, which is open sourced on Github as a repository
\footnote{\url{https://github.com/Microsoft/Recommenders}}.
The development of this repository is a {\em collaborative effort} of machine learning researchers and data scientists from 
various groups in Microsoft, as well as contributors from the community.
At the time of writing, the repository is the most popular one for recommendation systems on Github, with $\sim4800$ stars. The repository mainly consists of {\em utilities} for data manipulation, evaluation, 
model training and recommendation, as well as {\em Jupyter notebooks} containing how-to examples for building end-to-end recommenders, hands-on familiarization with the algorithms and quick prototyping. 
The repository is developed based on both industrial experience from the core team as well as the well-known libraries avaliable in the market.
The repository is designed with particular consideration on modularity and usability, so that custom utility functions and notebooks can be added with minimal ramp-up efforts.

The rest of the paper is organized as follows. Section 2 gives an overview of the structure and content of the repository. Section 3 explains design philosophy and how it benefits productive development for practitioners. Section 4 presents deployment and operationalization of recommender models with real-world industrial practice discussed. The paper is concluded in Section 5 with insights and learnings gained from the practical application of the recommendations repository.

