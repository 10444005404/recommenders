\section{Introduction}

Recommendation systems have become ubiquitous in modern business and have attracted a great deal of interest in academic research.
According to McKinsey \& Co,
“35\% of what consumers purchase on Amazon and 75\% of what they watch on Netflix come from recommendations algorithms" 
\cite{mckinsey}.
Recommendation appears almost everywhere in today's e-commerce, in which consumers typically have a large variety of products or other
options to choose from. Examples include recommendation of brands, news, prodcuts etc. The goal of business is usually to increase revenue,
customer engagement and satisfaction, induce networking effects etc. Another related goal may be to improve prediction of the metrics of interest, 
so that better business decisions about advertising spend and internet traffic can be made. In addition, vendors want to understand better their customers
or segments of customers in order to address them with more personalized and suitable marketing campaigns.

The history of recommendation algorithms goes back to the 1990s \cite{Tapestry, GroupLens}
with the first content based and collaborative filtering methods.
Around 2006 the Netflix prize \cite{} gave a boost to research in the area, in particular 
factorization-based models \cite{koren2009matrix,SVD++,PMF}.
Later on, more hybrid methods involving machine learning methods as well as factorization have appeared
\cite{FM,pairwise}.
In the past few years, deep learning methods have started to be applied to this area too
\cite{PNN,cheng2016wide,DeepFM,lian2018xdeepfm,he2017neural}.
Moreover, recommendation is a broad area of machine learning and includes a diverse set of tasks and goals, such as 
explainable recommendation \cite{}, knowledge enhanced recommendation \cite{},
reinforcement learning \cite{}, transfer learning \cite{} etc.

In practice, however, application of recommendation algorithms has found significant challenges. 
First, there is limited reference and guidance for building a recommendation system on scale to support
enterprise grade scenarios. In addition, off-the-shelf packages, tools and modules may be fragmented and not compatible with
each other. While there are new algorithms being published continuously in research, many practitioners may not have 
expertise or time to implement and deploy these algorithms in production.

In response to these challenges, our team has developed the Microsoft-Recommenders
Github repository (\url{https://github.com/Microsoft/Recommenders}).
The development of this repository is a {\em collaborative effort} of machine learning researchers and data scientists from the 
Azure Cloud and Microsoft Research organizations, as well as contributors from the outside academic researchers and data scientists.
Broadly speaking, it contains 
\begin{itemize}
\item
Utilities: modular functions for model creation, data manipulation, evaluation, etc.
\item
Algorithms: SVD, SAR, ALS, NCF, Wide\&Deep , xDeepFM , DKN,
\item
Notebooks: HOW TO examples for end to end recommender building.
\end{itemize}

In the following, we summarize the content of Microsoft-Recommenders and present the approach and the principles we have followed
for its development.
