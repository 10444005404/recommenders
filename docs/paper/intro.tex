\section{Introduction}

Recommendation systems have become ubiquitous in modern business, 
%and have attracted a great deal of interest in academic research.
%According to McKinsey \& Co,
%“35\% of what consumers purchase on Amazon and 75\% of what they watch on Netflix come from recommendations algorithms" 
%\cite{mckinsey}.
%Recommendations appear almost everywhere in today's e-commerce, in which consumers typically have a large variety of products or other
%options to choose from. 
examples of which include recommendation of brands, news, consumer products, media content, 
travel packages and many others. In such scenarios, the goals of business usually are to increase {\em revenue},
{\em customer engagement and satisfaction}, make better {\em predictions}, 
%so that better business decisions about advertising spend and internet traffic can be made. In addition, vendors want to 
gain {\em understanding about customers} etc.
%or segments of customers, in order to address them with more personalized and suitable marketing campaigns.
In machine learning research, the literature on recommendations goes back to the 1990s \cite{Tapestry, grouplens}
and includes a wide variety of approaches, such as {\em collaborative filtering} \cite{bell_lessons,koren2009matrix,SVD++,PMF}, 
{\em content-based} and {\em hybrid} methods \cite{rendle,ffm,bpr,pairwise,multiverse}, 
{\em deep learning} methods \cite{PNN,cheng2016wide,lian2018xdeepfm,he2017neural,youtube,nvidia,survival},
 {\em explainable recommendation} \cite{explainable,rl_explainable}, {\em reinforcement learning} \cite{rl_explainable,rl,rl_negative} etc.

%The history of recommendation algorithms goes back to the 1990s \cite{Tapestry, grouplens}
%when the first {\em content-based and collaborative filtering} methods appeared.
%Around 2006, the Netflix prize \cite{netflix} provided an additional boost to research in the area, in particular leading to significant advances in factorization-based models \cite{bell_lessons,koren2009matrix,SVD++,PMF}.
%Later on, more {\em hybrid methods} incorporating standard supervised learning into recommendations as well as more hybrid factorization models have appeared
%\cite{rendle,ffm,bpr,pairwise,multiverse}.
%In more recent years, {\em deep learning} methods, after their success in other areas of machine learning, have been applied increasingly in the area of recommendations
%\cite{PNN,cheng2016wide,lian2018xdeepfm,he2017neural,youtube,nvidia,survival}.
%Moreover, the objectives in recommendation problems are broader than, say, prediction accuracy or precision of recommendations, and include a diverse set of tasks and goals such as 
%{\em explainable recommendation} \cite{explainable,rl_explainable}, corrections for {\em biases} and offline {\em evaluation} \cite{mnar,offline,debiasing,counterfactual},
%{\em reinforcement learning} \cite{rl_explainable,rl,rl_negative} etc.

In practice, however, application of recommendation algorithms has encountered significant challenges. 
First, there are {\em limited references and guidance} for building recommendation systems at scale to support
enterprise grade scenarios. In addition, even though there exist several off-the-shelf packages, tools and 
modules, they tend to focus on specific aspects of recommendations 
and may {\em not be compatible} with each other. Whereas new algorithms are being published continuously in research, in the field many practitioners may lack the 
expertise or time to implement and deploy these algorithms in production. 
On the other side, researchers, when applying their methods to real world scenarios, may lack experience of the domain of interest
or awareness of the best practices with respect to data science and software engineering. Thus, it can be {\em time-consuming} to build a new recommendation system from an
algorithm, even when available as software, or to incorporate it into an existing recommendations pipeline. It can also be time-consuming to build a prototype of a new or existing algorithm and to 
{\em compare the performance} of different algorithms on the same recommendations task. 

In response to these challenges, our team has developed {\em Microsoft Recommenders}, an open-source 
Github repository available at \url{https://github.com/Microsoft/Recommenders}.
The development of this repository is an ongoing {\em collaborative effort} of machine learning researchers and data scientists from 
various groups in Microsoft, as well as contributors from the broader community.
This is also demonstrated by the usage of the repository, which has been growing fast with more than 4000 stars making it, 
at the time of writing, the most popular repository for recommendation systems on Github.%and 600 forks from different organizations all over the world. 

Broadly speaking, the repository mainly consists of {\em utilities} for data manipulation, evaluation, 
model training and recommendation; and {\em Jupyter notebooks}, that is, how-to examples for building end-to-end recommenders, hands-on familiarization with the algorithms and quick prototyping. 
The development approach we follow benefits from existing best practices and software libraries already avaliable in the recommendations community.
In addition, we have strived to make the repository modular and easy to understand and use, so that researchers and practitioners can easily build
new or customized algorithms, utilities and notebooks.

In the following sections, we summarize the content of {\em Microsoft Recommenders} and present the approach and the principles we have followed
for its development. In Section 2, we provide an overview of the structure and content of the repository. In Section 3, we explain our development approach, some of
the design decisions and show how practitioners are enabled to use and contribute to the repository productively. In Section 4, we discuss how recommendations
algorithms can be brought to operation and how they can be incorporated within end-to-end pipelines, based on our experience with real-life production systems. 
Finally, in Section 5, we conclude with some insights and learnings we have gained from the practical application of the recommendations repository.

