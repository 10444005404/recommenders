\subsection{Recommenders Notebooks}

The second main component of Microsoft Recommenders are the notebooks. They are examples and best practices, 
written in Jupyter notebooks, for building recommendation systems.
%NOTE: instead of citing the exact names of the folders that can change in the future, I list the high level concepts

There are notebooks for every part of a standard recommendation pipeline. In general terms the notebooks can be 
classified into the following categories:
\begin{itemize}
    \item \textbf{Quick start notebooks:} The quick start notebooks are examples of recommendation algorithms that 
    can be the run in less than 15min. The objective is to quickly introduce some examples.
    \item \textbf{Deep dive notebooks:} These notebooks contain a more detailed description of a system. In some cases
    we summarize the main mathematical development of an algorithm and show its implementation.
    \item \textbf{General notebooks:} They cover general use cases like data preparation, evaluation, hyperparameter
    tuning or operationalization.
\end{itemize}

