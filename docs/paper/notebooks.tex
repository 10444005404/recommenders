\subsection{Recommenders Notebooks}

The second main component of Microsoft Recommenders are the Jupyter notebooks. 
They demonstrate examples and best practices for building recommendation systems.
%NOTE: instead of citing the exact names of the folders that can change in the future, I list the high level concepts
There are notebooks for each part of a standard recommendation pipeline. They are
classified in the following categories:
\begin{itemize}
    \item \textbf{Quick start notebooks:} The quick start notebooks are examples of recommendation algorithms that 
    can be run in less than 15 minutes. The objective is to quickly introduce some examples and concepts.
    \item \textbf{Deep dive notebooks:} These notebooks contain a more detailed description of a system. In some cases
    we summarize the main mathematical development of an algorithm and show how it can be implemented with code.
    \item \textbf{General notebooks:} They cover general use cases like data preparation, evaluation, hyperparameter
    tuning or operationalization.
\end{itemize}

