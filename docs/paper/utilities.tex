\subsection{Recommenders Utilities}

The objective of the utilities is to facilitate and speed up the development and evaluation of recommendation systems.
These functions are designed as a the minimum layer needed for supporting common recommendation pipelines, some of 
them depicted in the notebooks. 

In some cases, the code addition is minimal, such is the case of the utilities on Surprise \cite{Surprise} or 
FastAI \cite{howard2018fastai} libraries. In some cases, full algorithms are provided, such is the case of 
some algorithms developed by Microsoft like SAR, RLRMC or the DeepRec package.

The components included in the utilities are in general terms the following: 
%NOTE: instead of citing the exact names of the folders that can change in the future, I list the high level concepts
\begin{itemize}
    \item \textbf{Common utils:} We provide supporting utilities like timers, loggers, gpu functions, 
    spark helpers or constants.
    \item \textbf{Data preparation:} We coded utility functions for data operations 
    such as data download, data loading, data transformation, data split, etc., which are frequent data preparation 
    tasks witnessed in recommendation system development. They include custom downloaders for some
    popular datasets like Movielens and Criteo and different types of data splitters like random, chronological or
    stratified.     
    \item \textbf{Algorithms:} We work in close collaboration with Microsoft Research for providing the latest state of
    the art algorithms like xDeepFM \cite{lian2018xdeepfm} or DKN \cite{wang2018dkn}, and some algorithms developed by other
    teams at Microsoft like SAR or RLRMC. There are contributions from the research community, a good case is NCF 
    \cite{he2017neural}, added by the authors of the algorithm. We provide some light wrappers of well-known Microsoft
    libraries like LightGBM \cite{ke2017lightgbm} or Vowpal Wabbit \cite{agarwal2014reliable}, or implementations
    of some algorithms like Wide and Deep \cite{cheng2016wide} or RBM \cite{salakhutdinov2007restricted}. 
    \item \textbf{Evaluation:} We implemented different rating and ranking metrics in Python+CPU and PySpark environments.
    \item \textbf{Model Selection and Optimization:} We added utility functions for a number of hyperparameter 
    tuning methods like NNI \cite{nni}. 
    \item \textbf{Operationalization:} We have operationalization functions on Kubernetes.
\end{itemize}
    

